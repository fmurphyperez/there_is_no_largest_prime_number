% Percent signs introduce comments. Please read all the comments.
% Compile with xelatex and use biber to generate the bibliography.
% Load the beamer class. 
\documentclass{beamer}
% Change some defaults.
\mode<presentation> % This will only apply to the presentation mode.
{
  \usetheme{Frankfurt} % The default theme is default
  % or metropolis
  % or SimplePlus
  \usecolortheme{seahorse} % and the default colortheme is default
  \usecolortheme{rose}  % Can use multiple colorthemes.
  % or ...beaver
  % or ...dolphin
  % or ...dove (more white)
  % or ...lily (gray bg)
  % or ...orchid
  % or ...rose
  % A piece of advice, not all themes can go with all colorthemes.
  % A nice place to check all combinations is the next url
  % https://hartwork.org/beamer-theme-matrix/
  \setbeamercovered{transparent}
}
% Now fill in the title & (subtitle).
\title[Short] % Optionally, write a short title between the brackets
{There Is No Largest Prime Number} 
% Short versions of title and subtitle will appear on the slides,
% depending on the theme used.
\subtitle[Short Subtitle]
{Long Subtitle}
% Now fill in the author(s).
% For one author, uncomment the next two lines and comment the lines 36-48:
%\author[Euclid]{Euclid of Alexandria \\ %\texttt{euclid@alexandria.edu}}
%\institute[University of Alexandria]{University of Alexandria}
% For multiple authors, uncomment lines 36-48 and comment the two lines above
\author[Author, Euclid]{Author\inst{1} \and Euclid of Alexandria\inst{2}}
% - Give the names in the same order as the appear in the paper.
% - Use the \inst{?} command only if the authors have different affiliation.
\institute[Universities of Somewhere and Alexandria] % Also optional.
{
  \inst{1}%
  Department of Computer Science\\ % This will make a return in the pdf.
  University of Somewhere
  \and
  \inst{2}%
  Department of Philosophy\\
  University of Alexandria
}
% - Use the \inst command only if there are several affiliations.
% - Keep it simple, no one is interested in your street address.
% Now fill in the date.
\date[ISPN ?80]{27th International Symposium of Prime Numbers}
% - Either use conference name or its abbreviation.
% - Not really informative to the audience, more for people (including
%   yourself) who are reading the slides online
% Now fill in the subject.
\subject{Prime Numbers}
% This is only inserted into the PDF information catalog. Can be left
% out.
% Now put the logo of your university or whatever.
% If you have a file called "university-logo-filename.xxx", where xxx
% is a graphic format that can be processed by latex or pdflatex,
% resp., then you can add a logo as follows:
% \pgfdeclareimage[height=0.5cm]{university-logo}{university-logo-filename}
% \logo{\pgfuseimage{university-logo}}
% Now ...
% Delete this, if you do not want the table of contents to pop up at
% the beginning of each subsection:
%\AtBeginSubsection[]
%{
  %  \begin{frame}<beamer>{Outline}
    %    \tableofcontents[currentsection,currentsubsection]
    %  \end{frame}
  %} % I do not find this really useful.

% If you wish to uncover everything in a step-wise fashion, uncomment
% the following command: 
%\beamerdefaultoverlayspecification{<+->} % So... we have to explain this
% For the slide 'What Are Prime Numbers?'
% if line 78 is uncommented, we will have a 1:5 (frame:slides)
% if line 78 is   commented, we will have a 1:1 ratio.
% In BEAMER, a presentation consists of a series of frames. Each frame in
% turn may consist of several slides (if there is more than one, they are
% called overlays).
% Now... you can finally start typing your presentation :)
\begin{document}
  
  \begin{frame}
    \titlepage % This generates the title page!
  \end{frame}
  
  \begin{frame}{Outline}
    \tableofcontents[pausesections] % This generates the table of contents!
    % Remove the option within brackets, recompile and see the difference
  \end{frame}
  
  \section{Motivation}
  
  \subsection{Changed to something more reasonable}
  
  \begin{frame}
    \frametitle{What Are Prime Numbers?}
    \begin{definition}
      A \alert{prime number} is a number that has exactly two divisors.
    \end{definition}
    \begin{example}
      \begin{itemize}
        \item 2 is prime (two divisors: 1 and 2).
        %\pause % Remove the comment (%), recompile and see the difference
        \item 3 is prime (two divisors: 1 and 3).
        %\pause % Remove the comment (%), recompile and see the difference
        \item 4 is not a primer (\alert{three} divisors: 1, 2, and 4).
      \end{itemize}
    \end{example} 
  \end{frame}
  \section{Results}
  \subsection{Somethin' else}
  \begin{frame}
    \frametitle{There Is No Largest Prime Number}
    \framesubtitle{The proof uses \textit{reductio ad absurdum}.}
    Use of \texttt{uncover} command.
    \begin{theorem}
      There is no largest prime number.
    \end{theorem}
    \begin{proof}
      \begin{enumerate}
        \item<1-> Suppose $p$ were the largest prime number. 
        \item<2-> Let $q$ be the product of the first $p$ numbers. \item<3-> Then $q + 1$ is not divisible by any of them. 
        \item<1-> But $q + 1$ is greater than $1$, thus divisible by some prime number not in the first $p$ numbers.\qedhere 
      \end{enumerate}
    \end{proof}
    \uncover<4->{The proof used \textit{reductio ad absurdum}.}
  \end{frame}
  
  \begin{frame}[t] % Remove the [t] option, recompile and see the difference
    \frametitle{There Is No Largest Prime Number}
    \framesubtitle{The proof uses \textit{reductio ad absurdum}.}
    Use of \texttt{only} command.
    \begin{theorem}
      There is no largest prime number.
    \end{theorem}
    \begin{proof}
      \begin{enumerate}
        \item<1-> Suppose $p$ were the largest prime number. 
        \item<2-> Let $q$ be the product of the first $p$ numbers. \item<3-> Then $q + 1$ is not divisible by any of them. 
        \item<1-> But $q + 1$ is greater than $1$, thus divisible by some prime number not in the first $p$ numbers.\qedhere 
      \end{enumerate}
    \end{proof}
    \only<4->{The proof used \textit{reductio ad absurdum}.}
  \end{frame}
  
  \begin{frame}
    \frametitle{What's Still To Do?}
    One option:
    \begin{block}{Answered Questions}
      How many primes are there?
    \end{block}
    \begin{block}{Open Questions}
      Is every even number the sum of two primes?
    \end{block}
  \end{frame}
  % Or like this?
  \begin{frame}
    \frametitle{What's Still To Do?}
    Another option:
    \begin{itemize}
      \item Answered Questions
      \begin{itemize}
        \item How many primes are there?
      \end{itemize}
      \item Open questions
      \begin{itemize}
        \item Is every even number the sum of two primes?
      \end{itemize}
    \end{itemize}
  \end{frame}
  % Or like this?
  \begin{frame}
    \frametitle{What's Still To Do?}
    Yet another option$\ldots$
    \begin{columns}[t]
      \column{0.5\textwidth}
      \begin{block}{Answered Questions}
        How many primes are there?
      \end{block}
      \pause
      \column{0.5\textwidth}
      \begin{block}{Open Questions}
        Is every even number the sum of two primes?
        \cite{Goldbach1742}
      \end{block}
    \end{columns}
  \end{frame}
  % Euclid normally uses the verbatim environment and sometimes also similar environments like lstlisting to typeset listings. He can also use them in beamer, but he must add the fragile option to the frame:
  \begin{frame}[fragile]
    \frametitle{An Algorithm For Finding Prime Numbers.}
\begin{verbatim}
  int main (void)
  {
    std::vector<bool> is_prime (100, true);
    for (int i = 2; i < 100; i++)
      if (is_prime[i])
      {
      std::cout << i << " ";
      for (int j = i; j < 100; is_prime [j] = false, j+=i);
      }
    return 0;
  }
\end{verbatim}
    \begin{uncoverenv}<2>
      Note the use of \verb|std::|.
    \end{uncoverenv}
  \end{frame}
  
  % Or like this...
  % the environment {semiverbatim} defined by beamer is more useful: It works 
  %like {verbatim}, except that \, {, and } retain their meaning (one can 
  %typeset them by using \\, \{, and \}).
  \begin{frame}[fragile]
    \frametitle{An Algorithm For Finding Prime Numbers.}
\begin{semiverbatim}
  \uncover<1->{\alert<0>{int main (void)}}
  \uncover<1->{\alert<0>{\{}}
    \uncover<1->{\alert<1>{\alert<4>{std::}vector<bool> is_prime (100, true);}}
    \uncover<1->{\alert<1>{for (int i = 2; i < 100; i++)}}
      \uncover<2->{\alert<2>{if (is_prime[i])}}
      \uncover<2->{\alert<0>{\{}}
      \uncover<3->{\alert<3>{\alert<4>{std::}cout << i << " ";}}
      \uncover<3->{\alert<3>{for (int j = i; j < 100;}}
      \uncover<3->{\alert<3>{  is_prime [j] = false, j+=i);}}
      \uncover<2->{\alert<0>{\}}}
    \uncover<1->{\alert<0>{return 0;}}
  \uncover<1->{\alert<0>{\}}}
    \end{semiverbatim}
    % The \visible command does nearly the same as \uncover. However one difference occurs if the command \setbeamercovered{transparent} has been used to make covered text “transparent” instead, \visible still makes the text completely “invisible” on non-specified slides.
    \visible<4->{Note the use of \alert{\texttt{std::}}.}
  \end{frame}
  
  % All of the following is optional and typically not needed. 
  \appendix
  \section<presentation>*{\appendixname}
  \subsection<presentation>*{For Further Reading}
  
  \begin{frame}[allowframebreaks]
    \frametitle<presentation>{For Further Reading}
    
    \begin{thebibliography}{10}
      
      \beamertemplatebookbibitems
      % Start with overview books.
      
      \bibitem{Author1990}
      A.~Author.
      \newblock {\em Handbook of Everything}.
      \newblock Some Press, 1990.
      
      
      \beamertemplatearticlebibitems
      % Followed by interesting articles. Keep the list short. 
      
      \bibitem{Someone2000}
      S.~Someone.
      \newblock On this and that.
      \newblock {\em Journal of This and That}, 2(1):50--100,
      2000.
      \bibitem{Goldbach1742}[Goldbach, 1742]
      Christian Goldbach.
      \newblock A problem we should try to solve before the ISPN '43 deadline,
      \newblock \emph{Letter to Leonhard Euler}, 1742.
    \end{thebibliography}
  \end{frame}
  
\end{document}


